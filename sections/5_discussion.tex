\documentclass[../main.tex]{subfiles}
\setlength{\parindent}{10pt}

\begin{document} \noindent
In this MEG study, we explored resonance and entrainment in the human visual system in response to a rapid photic drive $>$50 Hz. Strong, sustained gamma oscillations were induced using moving grating stimuli \citep{hoogenboom2006localizing,hoogenboom2010visually,van2013visual,muthukumaraswamy2013visual} \textcolor{red}{and used to identify each participant's gamma frequency. The superposition of the flicker and the gratings allowed us to investigate whether the flicker could entrain endogenous gamma oscillations.} The photic drive induced responses for frequencies up to $\sim$80 Hz, both in presence and absence of grating-induced endogenous gamma oscillations. To our surprise, we did not find evidence for resonance, i.e. an amplification of an individually preferred frequency in the range of the rhythmic stimulation, in either condition, despite the IGF being above 50 Hz in all participants. Moreover, there was no indication that the endogenous gamma oscillations synchronized with the rhythmic stimulation, i.e. no evidence for entrainment. \textcolor{red}{Despite their differences, brain activity in the two conditions show strong similarities in the phase and frequency measures, supporting the notion that the flicker response coexists with the grating-induced oscillations. In accordance with these results,} source estimation using Linearly Constrained Minimum Variance (LCMV) spatial filters \citep{van1997localization}, suggests that the neuronal sources of the flicker responses in both conditions and the endogenous gamma oscillations peak at different locations in visual cortex.

\subsection{Flicker responses do not entrain the gamma oscillator} \noindent
While the sources of the gamma oscillations and the response to the (nearly) invisible flicker did overlap in occipital cortex, their peak coordinates were found to be significantly different. \textcolor{red}{Relative power change at the IGF peaked at sources inferior to the flicker responses in both conditions, and was located in the left secondary visual cortex (V2) using the MNI2TAL online tool \citep[see][]{lacadie2008brodmann,lacadie2008more}. The flicker peak sources were located in the Calcarine Fissure, in close proximity to the primary visual cortex (V1). These results are in line with the coexistence of the endogenous oscillations indicated by the time-frequency analyses and might be the result of the filter properties of synaptic transmission as the flicker response propagates in the visual system \citep[see][]{kuffler1953discharge,hawken1996temporal,carandini1997linearity,ringach2004mapping,cormack2005computational,shadlen1999synchrony}.} Low-pass filtering at the transition from the thalamus to V1 \citep{connelly2016thalamus} might attenuate the photic drive at frequencies above 80 Hz, leading to an absence of measurable responses in this range. Low-pass filter properties in V1 in projections from granular layers (L4a, 4c$\alpha$ and 4c$\beta$)  to supragranular  (L2/3, 4b) and infragranular layers (L5,6) \citep{hawken1996temporal,douglas2004neuronal,frohlich2016network} might have prevented the flicker response to converge to the neuronal circuits generating the endogenous gamma rhythms. This idea is supported by intracranial recordings in macaques showing the strongest gamma synchronization in response to drifting grating stimuli in V1 in supragranular layers (L2/3 and 4B) \citep{xing2012laminar}, whereas steady-state responses to a 60 Hz photic flicker were localised in granular layer 4c$\alpha$ \citep{williams2004entrainment}. \textcolor{red}{While plausible, these interpretations are conjectural based on the present data.} Recent findings by \citet{drijvers2020rapid}, providing evidence for non-linear integration of visual and auditory rapid frequency tagging signals in frontal and temporal regions, challenge \textcolor{red}{the notion that the flicker response might not propagate beyond V1.} Pairing the current paradigm with intracranial recordings in non-human primates would allow to test the filtering properties without the limitations imposed by the inverse problem in the source localization of neuromagnetic signals \citep{baillet2013forward}.
\paragraph{Flicker responses might not be wired to inhibitory interneurons orchestrating the endogenous gamma rhythm}
Computational models, as the one demonstrated by \citet[][also see \citealp{lee2013distinguishing}]{tiesinga2012motifs}, would be suitable to investigate whether the grating-induced gamma oscillations and flicker response are likely to be generated by neuronal circuits whose wiring is not conducive to entrainment. As the properties of neuronal gamma oscillations have been repeatedly shown to depend on rhythmic inhibition imposed by inhibitory interneurons \citep[e.g.][]{wilson1972excitatory,bartos2007synaptic,buzsaki2012mechanisms,lozano2014gabaergic,kujala2015gamma}, entrainment should only be achieved when the flicker response is able to modulate their activity. Indeed, \citet{cardin2009driving} show resonance in the gamma range to optogenetic stimulation of fast-spiking interneurons, but not to stimulation of pyramidal cells \citep[also see][]{tiesinga2012motifs}. We therefore suggest that the photic stimulation applied in our study drives the pyramidal cells in early visual cortex. As in the optogenetic study by \citet{cardin2009driving}, this drive is not sufficiently strong to entrain the GABAergic interneurons. 
This interpretation is contrasted to the findings of  \citet{adaikkan2019gamma2} who demonstrate that a non-invasive 40 Hz flicker evokes neuronal processes counteracting neuro-degeneration \citep{singer2018noninvasive,adaikkan2019gamma2}. However, it should be noted that the authors understand entrainment as the neural response to rhythmic stimulation, rather than a synchronization of ongoing oscillations to an external drive \citep{adaikkan2020gamma}. While our findings do not question the authors' compelling evidence that fast photic stimulation impacts neurocircuits and glia, the current study shows that it is not trivial to attribute these effects to entrainment of endogenous gamma oscillations.
\subsection{\textcolor{red}{Coexistence of flicker responses and oscillations versus oscillatory entrainment}} \noindent
\textcolor{red}{The current study was inspired by studies reporting that a visual flicker in the alpha-band can capture the oscillatory dynamics of the visual system: resonance at distinct frequencies \citep[][see \citealt{rager1998response} for flicker responses in cat visual cortex]{herrmann2001human,schwab2006alpha,gulbinaite2019attention},  amplitude and phase effects outlasting the stimulation interval \citep{spaak2014local, otero2020persistence} and an "Arnold Tongue" relationship between stimulation intensity, distance to the individual alpha frequency and flicker-response-synchrony \citep{notbohm2016modification}. Unlike the works listed above, we did not find any indication for a synchronization or resonance of endogenous oscillations in the gamma band to the visual stimulation. Recent studies applying photic stimulation in the alpha band, have pointed to a coexistence of endogenous alpha oscillations and flicker responses, similar to the one we report here for the gamma band. While retinotopic alpha modulation has been associated with suppression of unattended stimuli, allocating attention to a stimulus flickering in the alpha band results in enhanced, phase-locked activity \citep[][also see \citealt{antonov2020too,friedl2020effects} for stimulation at frequencies adjacent to the alpha-band]{keitel2019stimulus,gundlach2020spatial}. While the presented study does not allow nor aim to make generalized claims in favor or against neuronal entrainment, it is worth noting that the ability of rhythmic sensory stimulation to entrain endogenous oscillations is still a matter of debate.} 
\textcolor{red}{\subsection{Limitations \& Generalizability}\noindent
\paragraph{Interpretation of the different locations of the peak sources}
The results of the LCMV beamforming are in line with the notion that gamma oscillations and flicker response are generated by sources at different locations. Yet, due to the ill-posed inverse problem \citep{baillet2013forward} and the merging of coherent sources when using the LCMV approach \citep{belardinelli2012source} these source estimates should be interpreted with caution. Figure \ref{fig:sourceGA} illustrates that the sources of the flicker response in the \gammaRFT condition extended more broadly over visual cortex than the sources of the gamma oscillations and invisible flicker response, which might be the result of the flickering rings stimulating different receptive fields \citep{gur1997visual}. While our results suggest a coexistence of the gamma oscillations and flicker response, we do not exclude they interact. These limitations do not seriously challenge our interpretation that the neuronal populations generating the flicker response do not entrain the activity of the neurons generating the endogenous gamma rhythm. Firstly, it is reasonable to assume that the peak sources reflect the flicker response, which tends to be stronger than the endogenous gamma oscillations (see Figure \ref{fig:subj_ent} and \ref{fig:TFRGA_ent}). Secondly, the significant difference between the peak locations of the gamma oscillations and flicker response in the \gammaRFT condition provides circumstantial evidence for the notion that the two responses emerge from different neuronal populations, despite being elicited by the same stimulus; albeit there is also overlap between the sources. Intracranial recordings in nonhuman primates or humans would be useful to substantiate this interpretation. 
\paragraph{Strong flicker responses despite limited stimulation strength}
The number of conditions that have been tested in this paradigm, i.e.  40 frequency$\times$condition combinations, imposed limitations on the maximum number of trials per condition (N=15) and the duration of the stimulation (2 seconds). Stimulation strength was limited to a contrast of 66\% peak to trough, ensuring equal luminance across conditions. Due to these limitations, one might be concerned that the absence of oscillatory entrainment was caused by the limited magnitude of the photic drive. However, we found the flicker to induce strong responses of up to 400\% in the \gammaRFT condition and over 200\% in the flicker condition (e.g. see Figures \ref{fig:av_res} and \ref{fig:freq_fun}). In light of these response magnitudes, we argue that the absence of evidence for entrainment cannot be explained by the photic drive being too weak.}
\paragraph{Generalizability of the current findings to gamma oscillations associated with visual perception}
\textcolor{red}{The use of drifting gratings is a standard approach to induce strong narrow-band gamma oscillations in humans \citep[e.g.][]{hoogenboom2006localizing,hoogenboom2010visually,muthukumaraswamy2013visual,van2012magnetoencephalography,van2013visual,michalareas2016alpha} and nonhuman primates \citep[e.g.][]{womelsdorf2006gamma,bosman2012attentional,buffalo2011laminar}.}
One might argue that the conclusions presented here only apply to these stimuli and that entrainment could have been achieved using more complex stimuli such as natural images or faces. We find this very unlikely for the following reasons:
Natural stimuli have been argued to induce gamma-band responses that are characterized by broadband activity \citep[][but also see \citealt{brunet2014gamma,bartoli2019functionally,brunet2019human}]{ray2010differences,hermes2015gamma,hermes2015stimulus}. This is likely explained by the fact that gamma power and frequency depend on stimulus properties such as contrast, size and orientation \citep{schadow2007stimulus,ray2010differences,jia2013no,muthukumaraswamy2013visual}. As these factors vary greatly within a natural image, the net result of the oscillatory activity in the gamma-band is a broadband response.  \textcolor{red}{Moving gratings have been shown to induce stronger gamma oscillations than their stationary counterparts \citep{muthukumaraswamy2013visual,perry2013properties} and were therefore chosen for the current paradigm. We expected the flicker responses to be substantially stronger than the grating-induced gamma oscillations, which is confirmed by Figure \ref{fig:subj_ent} and \ref{fig:TFRGA_ent}. Had we relied on stationary gratings, the photic drive might have overshadowed weaker gamma-band activity. Moreover, the frequencies of the endogenous gamma rhythms have been found to be higher for moving than for stationary gratings \citep{muthukumaraswamy2013visual,perry2013properties}. As our study aimed to investigate entrainment by a flicker with minimal visibility, the IGFs had to be relatively high to be in the range of feasible stimulation frequencies. While the gratings' concentric drift did induce a rhythmic response at 4 Hz, there was no evidence for an intermodulation with the flicker frequencies, nor an indication that the \gammaRFT condition was lacking spectral precision.}
Another concern might be that grating stimuli do not engage downstream regions to the same extent as complex stimuli; as such they might be generated in specialized neuronal circuits. However, a number of studies in both human and non-human primates have demonstrated that attended as well as unattended gratings induce gamma oscillations that propagate to downstream areas along the ventral (V4 and inferotemporal cortex) and dorsal stream (area V5 and V7) \citep{buffalo2011laminar,bosman2012attentional,bastos2015visual,michalareas2016alpha}.
For the reasons outlined above, we argue that moving grating stimuli created the optimal conditions to investigate gamma-band entrainment, as these induced strong, sustained, narrow-band gamma oscillations reflecting individual oscillatory dynamics \citep[also see][]{hoogenboom2006localizing,van2013visual}.
\subsection{Concluding remarks} \noindent
Our results suggest that rapid photic stimulation does not entrain endogenous gamma oscillations and can therefore not be used as a tool to probe the causal role of gamma oscillations in cognition and perception. However, the approach can be applied as  Rapid Frequency Tagging (RFT) to track neuronal responses without interfering, for instance, to investigate covert spatial attention \citep{zhigalov2019probing}, multisensory integration \citep[][]{drijvers2020rapid} and parafoveal reading \citep{pan2020lexical}.


\end{document}